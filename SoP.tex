\documentclass{article}
\usepackage[letterpaper,margin=2in]{geometry}
\usepackage{xcolor}
\usepackage{fancyhdr}
\usepackage{tgschola} % or any other font package you like

\pagestyle{fancy}
\fancyhf{}
\fancyhead[C]{%
  \footnotesize\sffamily
  \yourname\quad
  web: \textcolor{blue}{\itshape\yourweb}\quad
  \textcolor{blue}{\youremail}}

\newcommand{\soptitle}{Statement of Purpose}
\newcommand{\yourname}{Katsuya Fujii}
\newcommand{\youremail}{katsuyaf@media.mit.edu}
\newcommand{\yourweb}{http://www.katsuyaf.com/}

\newcommand{\statement}[1]{\par\medskip
  \underline{\textcolor{black}{\textbf{#1:}}}\space
}

\usepackage[
  colorlinks,
  breaklinks,
  pdftitle={\yourname - \soptitle},
  pdfauthor={\yourname},
  unicode
]{hyperref}

\begin{document}

\begin{center}\LARGE\soptitle\\
\large of \yourname\ (Massachusetts Institute of Technology Fall---2015)
\end{center}

\hrule
\vspace{1pt}
\hrule height 1pt

\bigskip

%You should describe why you wish to attend graduate school, what you would like to study, and any research experience you have. Describe one or more accomplishments you are particularly proud of that suggest that you will succeed in your chosen area of research.

I am applying for Media Arts and Science, MIT Media Lab, at the Massachusetts Institute of Technology and wish to specialize in the Human Computer Interaction (HCI). 
I am very attracted by MIT Media Lab's unique culture:  "Deploy or Die", holding the best network across discipline and strongly academic yet entrepreneurial culture. 

%I have been engaging myself in broadening a wide variety of understanding of HCI, obtaining knowledge expanding from mechanical engineering, electrical to computer science skills and enriching leadership skill on collaborative working through team experiences. With those strong abilities,I strongly believe that I will fit well into the program and I want to use graduate study to expand my knowledge, gain more experience, and after that go into the industry and start my own company.

%My undergraduate degree in Media Informatics and work experience have given me adequate preparation and have formed my decision on how I wish to plan my future. 

\statement{Why I wish to attend graduate school} 
In order to translate ideas into impact, fabricating and manufacturing are essential. In this sense, brainstorming, prototyping and deploying take the most important roles to lead us to innovations and they are curved strongly in Media Lab's spirit like "Deploy or Die". This culture makes it unique to study at MIT Media Lab. I have been fascinated by this culture and I've been enriching my skills to be capable of doing so. Another unique aspect of MIT Media Lab they put emphasis on working across discipline. I have broaden strong understanding of HCI aiming to be a successful researcher. Yet my knowledge still needs to be elaborated. But most importantly, what we can benefit from this culture, is that it offers a fantastic environment to create a connection network of people. Last but not least, I always had an interest in entrepreneurship. MIT Media Lab offers a wide variety of support for entrepreneurship and several spin-off companies have been created.If accepted by this program, I'd like to expand my knowledge, elaborate my ideas through collaborative work and make an impact to our society with my future technology. 

The cost of innovation is dramatically decreasing,, says Joi ito, the director of MIT Media Lab. To be a successful person in academically or in industry, it is required that you have abilities to fabricate things, try them, and evaluate them. And what is happening these days is that now it is possible to translate ideas into impact all by yourself at incredibly low cost.  I've been also putting emphasis on enriching these skills and have been trying to have a variety of experiences. For instance,  as my background expands from software to hardware development, I believe I have a strong rapid-prototyping skill.  I have experiences in developing a management software, mobile applications, or other software developments through which I've learn how to write a reliable code for both front and server end and I know how to make them deployable in a real market. I've also worked on flying drones, wearable devices, social-interactive system, in which I trained some rapid-prototyping skills like 3D modeling, laser cutting as well as PCB design and I know how to evaluate their scalability . With the culture of Media Lab where various classes are designed to provide a hands-on introduction to the resources for designing and fabricating smart systems, I believe that can gain further knowledge of these skills and learn the spirit to make an impact in real world.

 Through research projects and classes, MIT Media Lab offers antidisciplinary and highly collaborative work with talented people from all over the world. Working on across discipline makes it possible for us to get a different perspective, give you an opportunity to realize what you are capable of or what you are not, and find that team working is a key to lead us to innovations. And more importantly you can create a network of people through these activities.
 
MIT Media lab culture also helps for my entrepreneurship. In 2011, I took the jump and started up an mobile application company that develops an app generator(iOS and Android) for music bands encouraging their promotion. I engaged myself as a co-founder and a leading programmer and developed both client and server side systems. It was a successful venture, but I soon realized it was very time-consuming and I was neglecting my studies. So, after two years, I decided to close it down and re-focus on my degree. In the process I learned not only leadership but also a lot about management and how businesses work. Even after I closed it down, my will to start up my own company has not been changed. To be more specific, one of my dreams is to aim for Tokyo Olympic 2020 hoping to introducing my technology. MIT Media Lab hosts a lot of collaborative opportunity with sponsor companies, many are Japanese. As a bridge between Japan and MIT Media Lab, I'd like to pursue my opportunity to realize my dream.

As I mentioned above, I believe my will, interest and dream will make me the best fit to the program.


%My first encoutner with HCI dates back to 2011, that was when I was doing an internship as a software developer in Spain. At that time I majored in in Electrical Engineering at college specializing electrical hardware development and semi-conductor. Thus I was not a great programmer ,had no knowleadge in Spanish and it was my first experience to work in an international environment. However, I put my maximum effort on adapting myself to new things and by the end of the program I managed to develop two different softwares from scratch. One of the softwares that I developed was for Project Management and this software was adopted as one of the main management softwares at a company that I worked for and still currently in use by a lot of people. This experience gave me a confidence to challenge new things without being afraid and nothing is impossible if you have a strong desire to "learn". I then made up my mind to change my major from Electrical Engineering to Computer Science, which would eventually give me a wide variety of knowleage ranging from harware to software.  While I was doing some research, I came across this TED video, called "SixSense",conducted by MIT Media Lab Fluid Interfaces Group. I clearly remember that as if I got struck by lightning I immediately decide to lead myself to this field called Human Computer Interaction. 

%As the first stepping stone to the world of HCI, I decided to attend my home country's University ,the University of Tokyo with the supervision of Prof. Jun Rekimoto. Prof.Jun is a world-widely well known researcher by his remarkable works in HCI. Even thought that was my first experience in HCI, working with Jun made it possible for me to do research at the cutting edge environment. 

%As I already received a degree in the same field, I have a strong understanding on HCI which helps me to conduct deeper research. My willingness to attend this program is not only just for a degree but also to establish new connections through antidisciplinary and highly collaborative work.


\statement{What I would like to study } Through my experiences, I notice the importance of learning. Leanring can fullfill your desire and brings you new opportunity. Currently, my interest in HCI is about learning. Currently I have two projects going on and the Fluid Interfaces Group has worked on the similar concept. This fact will encourage us to conduct further collaborative works.

"Realtime Task Teaching/Learning System using Haptic Feedback"
The idea is to remotely connect an expert and a beginner via 2 haptic robots to help the beginner learn manual tasks. A master in some task can in real time guide and/or correct an apprentice in a given task (while still giving them some freedom). Actual task applications for this system are to be discussed but I'm thinking about kitchen knife technique or 3D sculpting. In Fluid Interfaces Group. Judith and Xavier worked on project called Show me. The main concept is pretty similar to my idea:: Immersive mobile collaboration system that allows users to communicate remotely. This project will develop wearable technologies to capture the expressions of hands and non-invasive actuation technologies to transfer these expressions to others to enabling learning of new hand skills. Augmenting hands will allow for the safeguarding of ancient crafts and skills as well as enabling remote experts to aid and teach people at a distance. We record human knowledge today through text, images and video; this project will enable recording the nuances and precise skills, motions and expressions of the human hand through a wearable device. We will develop prototypes of this technology and present a series of innovative applications.


"Computer Supported Errorless Leaning" 


I'd like to introduce this new method for learning to the HCI community. Error-less Learning is a method introduced by B.F Skinner in 1930, who said that errors sometimes are not necessary for task learning. It is the opposite of the conventional "Trial and Error" method. I'd like to take advantage of computer technology so that users can rely on them and they won't make mistakes when they try to conduct a task. Users learn tasks by repeating correct movement guided by a computer. As they can learn tasks from successful experience, it possibly can help them  maintain motivation for task learning. The specific tasks I am looking at are things such as balancing a long stick on the tip of your finger. In Fluid Interfaces Group, Roy and Amit have been exploring the similar field. 

Free-D allows unskilled makers to produce complex carving tasks, as well as personalizing and modifying the digital 3D models while physically carving. The control software offers guidance according to static virtual models or dynamic ones, which may be altered directly or parametrically. In addition, the FreeD is also able to semi-autonomously move and carve. This creates synergistic cooperation between human and machine that ensures accuracy in recreation of the model while preserving the expressiveness of manual carving. 

Digital Air brash acts both as a physical spraying device and an intelligent digital guiding tool, that maintains both manual and computerized control.


\statement{Research experience and Accomplishments}
As my master project at the University of Tokoy, I first engaged in an interaction system with a flying drone.  
As a result of my project, I got accepted by two proceeding international conferencs as a full paper(Ubiquitous Intelligence and Computing Conference 2013 Dec and International Conference on Artificial Reality and Telexistence Conference 2013 Dec.) as well as a journal (IPS Journal Special issue of "Regional Contribution and Reconstruction", 2013). Alongside the school research, I worked at Sony Computer Science Laborarory as a reserach assistant. Like above, I have been making a great effort in order to be a succesful researcher. 

I have seveal experiences to prove my leadership skill. I regard myself as a curious and challenging person and I’ve been attetnding a lot of events such as organizing a work- shop, hackathon or class projects that required me to work as a team. I’ve always taken a leadership in those opporunities and most of the time I’ve brought the best result to our team, like awarding the first prize or having a research paper accepted by a conference..In 2013, I organized a workshop event held at MIT Media Lab and took a role as a director. We successed to gather over 100 people across the world for the event.I've got awared the best prize in every hackathon that I've attended, won the first prize at the Web-Promotaion Gran Prix in 2012 held by Atlas where we developed a travel agency advertiment and in 2014 won the first proce at the Me++ Hackathon, where we developed shoe-shaped wearable shoes that interact with a city that you live in. 

In addition to that ,I always had an interest in entrepreneurship. I took the jump and started up an mobile application company that develops an app generator(iOS and Android) for music bands encouraging their promotione. I engaged myself as a co-founder and a leading programmer and developed both client and server side systems. It was a successful venture, but I soon realized it was very time-consuming and I was neglecting my studies. So, after two years, I decided to close it down and re-focus on my degree. In the process I learned not only leadership but also a lot about management and how businesses work.


\statement{Outlook}

After graduate school, I wish to go into the industry, so I believe a master's degree is the ideal choice. I am looking forward to collaborating with the top professors in the field and I hope that with my background and the knowledge I will gain at MIT Media Lab I can make worthwhile contributions in the future.
\end{document}