\documentclass{article}
\usepackage[letterpaper,margin=2in]{geometry}
\usepackage{xcolor}
\usepackage{fancyhdr}
\usepackage{tgschola} % or any other font package you like

\pagestyle{fancy}
\fancyhf{}
\fancyhead[C]{%
  \footnotesize\sffamily
  \yourname\quad
  web: \textcolor{blue}{\itshape\yourweb}\quad
  \textcolor{blue}{\youremail}}

\newcommand{\soptitle}{Statement of Purpose}
\newcommand{\yourname}{Katsuya Fujii}
\newcommand{\youremail}{katsuyaf@media.mit.edu}
\newcommand{\yourweb}{http://www.katsuyaf.com/}

\newcommand{\statement}[1]{\par\medskip
  \underline{\textcolor{black}{\textbf{#1:}}}\space
}

\usepackage[
  colorlinks,
  breaklinks,
  pdftitle={\yourname - \soptitle},
  pdfauthor={\yourname},
  unicode
]{hyperref}

\begin{document}

\begin{center}\LARGE\soptitle\\
\large of \yourname\ (Massachusetts Institute of Technology Fall---2015)
\end{center}

\hrule
\vspace{1pt}
\hrule height 1pt

\bigskip

%You should describe why you wish to attend graduate school, what you would like to study, and any research experience you have. Describe one or more accomplishments you are particularly proud of that suggest that you will succeed in your chosen area of research.

I am applying for Media Arts and Science, MIT Media Lab, at the Massachusetts Institute of Technology and wish to specialize in the Human Computer Interaction (HCI). I am very attracted by MIT Media Lab's unique culture such as antidisciplinary, highly collaborative and strongly academic yet entrepreneurial culture. I have been engaging myself in broadening a wide variety of understanding of HCI, obtaining knowledge expanding from software and hardware skills and richening leadeship skill on collaborative working through experiences. With those strong abilities,I strongly beleive that I will fit well into the program and I want to use graduate study to expand my knowledge, gain more experience, and after that go into the industry and start my own company.

%My undergraduate degree in Media Informatics and work experience have given me adequate preparation and have formed my de- cision on how I wish to plan my future. 

\statement{Why I wish to attend graduate school} The first time that I met HCI dates back to 2011, that was when I was doing an internship as a software developer in Spain. As, I majored in in Electrical Engineering at college, to be honest, I still was not a great programmer.However, I put my maximum effort on finalizing tasks and managed to develop two different softwares from scratch. This experience gave me a confidence to challenge new things without being afraid of them. I then made up my mind to change my major from Electrical Engineering to Computer Science, which would eventually give me a wide variety of knowleage ranging from harware to software.  While I was doing some research, I came across this TED video, called "SixSense",conducted by MIT Media Lab Fluid Interfaces Group. I clearly remember that as if I got struck by lightning I immediately decide to lead myself to this field called Human Computer Interaction. 

As the first stepping stone, I decided to attend my home country's University ,the University of Tokyo with the supervision of Prof. Jun Rekimoto. Prof.Jun is a world-widely well known researcher by his remarkable works in HCI. Even thought that was my first experience in HCI, working with Jun made it possible for me to do research at the cutting edge environment. I stared with broadening my understanding of HCI by taking some courses and spending time on checking up some major conferences like CHI, SIGGRAPH, UIST or Ubicomp. I've done a lot of discussions with my colleagues and there I found myself good at filtering promising ideas out of hundreds of silly ideas. As a result, in these 2 years of my master course, I got accepted by two international conferences as well as a journal. I also got an opportunity to work as a research assistant at Sony Computer Science Laboratory. 


\statement{What I would like to study } Currently, my interest in HCI is abou learning. Currently I have two projects going on:
"Realtime Task Teaching/Learning System using Haptic Feedback"
The idea is to remotely connect an expert and a beginner via 2 haptic robots to help the beginner learn manual tasks. A master in some task can in real time guide and/or correct an apprentice in a given task (while still giving them some freedom). Actual task applications for this system are to be discussed but I'm thinking about kitchen knife technique or 3D sculpting. In Fluid Interfaces Group. Judith and Xavier worked on project called Show me. The main concept is pretty similar to my idea:: Immersive mobile collaboration system that allows users to communicate remotely. This project will develop wearable technologies to capture the expressions of hands and non-invasive actuation technologies to transfer these expressions to others to enabling learning of new hand skills. Augmenting hands will allow for the safeguarding of ancient crafts and skills as well as enabling remote experts to aid and teach people at a distance. We record human knowledge today through text, images and video; this project will enable recording the nuances and precise skills, motions and expressions of the human hand through a wearable device. We will develop prototypes of this technology and present a series of innovative applications.


"Computer Supported Errorless Leaning" 
I'd like to introduce this new method for learning to the HCI community. Error-less Learning is a method introduced by B.F Skinner in 1930, who said that errors sometimes are not necessary for task learning. It is the opposite of the conventional "Trial and Error" method. I'd like to take advantage of computer technology so that users can rely on them and they won't make mistakes when they try to conduct a task. Users learn tasks by repeating correct movement guided by a computer. As they can learn tasks from successful experience, it possibly can help them  maintain motivation for task learning. The specific tasks I am looking at are things such as balancing a long stick on the tip of your finger. In Fluid Interfaces Group, Roy and Amit have been exploring the similar field. 

Free-D allows unskilled makers to produce complex carving tasks, as well as personalizing and modifying the digital 3D models while physically carving. The control software offers guidance according to static virtual models or dynamic ones, which may be altered directly or parametrically. In addition, the FreeD is also able to semi-autonomously move and carve. This creates synergistic cooperation between human and machine that ensures accuracy in recreation of the model while preserving the expressiveness of manual carving. 

Digital Air brash acts both as a physical spraying device and an intelligent digital guiding tool, that maintains both manual and computerized control.


\statement{Research experience and Accomplishments}
In my master project, I got accepted by two proceeding international conferencs as a full paper(Ubiquitous Intelligence and Computing Conference 2013 Dec and International Conference on Artificial Reality and Telexistence Conference 2013 Dec.) as well as a journal (IPS Journal Special issue of "Regional Contribution and Reconstruction", 2013). Alongside the school research, I worked at Sony Computer Science Laborarory as a reserach assistant. Like above, I have been making a great effort in orde to be succesful as a researcher. 

I have seveal experiences to prove my leadership skill. I regard myself as a curious and challenging person and that leads myself to attend various events like workshops or competitions.In 2013, I organized a workshop event held at MIT Media Lab and took a role as a director. We successed to gather over 100 people across the world for the event. In addition to that,I'm fond of attending a Hackathon events. I've got awared the best prize in every hackathon that I've attended, won the first prize at the Web-Promotaion Gran Prix in 2012 held by Atlas where we developed a travel agency advertiment and in 2014 won the first proce at the Me++ Hackathon, where we developed shoe-shaped wearable shoes that interact with a city that you live in. In addition to that ,I always had an interest in entrepreneurship. I took the jump and started up an mobile application company that develops an app generator(iOS and Android) for music bands encouraging their promotione. I engaged myself as a co-founder and a leading programmer and developed both client and server side systems. It was a successful venture, but I soon realized it was very time-consuming and I was neglecting my studies. So, after two years, I decided to close it down and re-focus on my degree. In the process I learned not only leadership but also a lot about management and how businesses work.




I've been working on Human Computer Interaction for several years. It is often said that automation technologies such as auto-translating or auto-driving  system have a potential todeterioate human abilities. In stead of letting computers replace human abiity, I'd like to create a future where computer helps us learn new tasks, enhance human abilities. Error-less learning is a method that I'd like to introduce as a new way of task learning. As it literally says, people wont make mistakes when they learn a new task as opposed to a conventional trial and error learning method. Error-les learning provide us with a physical support and an optimal environment for task learning. By letting users havr sucessful xperience, it makes it efficient for us to learn a task and moreover it helps us maintain a motivation. I belive that realizing this concept  has a potential to contribute to HCI community. Fluid Interfaces group has been fucusing on learning by using technology. 

\statement{Conclusion}
In my bachelor's and master's course, I majored in Electrical Engineering and Computer Science respectively,  thus my skill expands from hardware to software development.
One of my great strengths is communicating and leadership skill, especially in a global environment. I've been attetnding a lot of events such as organizing a workshop, hackathon or class projects that required me to work as a team. I've always taken a leadership in those opporunities and most of the time I've brought the best result to our team, like awarding the first prize or having a research paper accepted by a conference. 
After graduate school, I wish to go into the industry, so I believe a master’s degree is the ideal choice. I see my future work in areas such as the interactive use of graphics, like real-time rendering in animation and especially video games, and creating therefore I believe working with the people from the Graphics, Games & Multimedia Research Area, especially the GamePipe Lab, will be very beneficial for me. I am looking forward to collaborating with the top professors in the field and I hope that with my background and the knowledge I will gain at USC I can make worthwhile contributions in the future.
\end{document}